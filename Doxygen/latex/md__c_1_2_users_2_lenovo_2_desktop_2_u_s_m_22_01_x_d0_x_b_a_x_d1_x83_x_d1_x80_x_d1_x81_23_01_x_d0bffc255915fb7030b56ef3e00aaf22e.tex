\chapter{Tic-\/\+Tac-\/\+Toe}
\hypertarget{md__c_1_2_users_2_lenovo_2_desktop_2_u_s_m_22_01_x_d0_x_b_a_x_d1_x83_x_d1_x80_x_d1_x81_23_01_x_d0bffc255915fb7030b56ef3e00aaf22e}{}\label{md__c_1_2_users_2_lenovo_2_desktop_2_u_s_m_22_01_x_d0_x_b_a_x_d1_x83_x_d1_x80_x_d1_x81_23_01_x_d0bffc255915fb7030b56ef3e00aaf22e}\index{Tic-\/Tac-\/Toe@{Tic-\/Tac-\/Toe}}
\label{md__c_1_2_users_2_lenovo_2_desktop_2_u_s_m_22_01_x_d0_x_b_a_x_d1_x83_x_d1_x80_x_d1_x81_23_01_x_d0bffc255915fb7030b56ef3e00aaf22e_autotoc_md0}%
\Hypertarget{md__c_1_2_users_2_lenovo_2_desktop_2_u_s_m_22_01_x_d0_x_b_a_x_d1_x83_x_d1_x80_x_d1_x81_23_01_x_d0bffc255915fb7030b56ef3e00aaf22e_autotoc_md0}%
{\bfseries{Описание задачи\+:}} Репозиторий содержит реализацию классической игры "{}Крестики-\/нолики"{} на C++. Игра предоставляет возможность двум игрокам сразиться в увлекательном соревновании, ставя крестики и нолики на игровом поле.

{\bfseries{Особенности\+:}}

Реализовано текстовое взаимодействие с игроками через консольный интерфейс. Игра поддерживает двух игроков. Проверка на победу и ничью. Простая и интуитивно понятная игровая механика.

{\bfseries{Описание правил игры\+:}}

"{}Крестики-\/нолики"{} -\/ это стратегическая настольная игра для двух игроков. Цель каждого игрока -\/ заполнить горизонталь, вертикаль или диагональ игрового поля своими символами (крестиками или ноликами) раньше соперника. Игроки поочередно делают ходы, ставя свой символ в пустую ячейку. Побеждает тот, кто первым создаст линию из трех своих символов.

{\bfseries{Зависимости проекта\+:}}


\begin{DoxyItemize}
\item {\bfseries{Среда разработки\+:}} Проект предполагает использование компилятора g++.
\item {\bfseries{Используемый компилятор\+:}} Mingw-\/get.
\item {\bfseries{Использованные сторонние библиотеки\+:}} Нет явных указаний на использование сторонних библиотек.
\end{DoxyItemize}

{\bfseries{Описание способа сборки приложения\+:}}

Для сборки проекта предоставлен скрипт {\ttfamily build.\+bat}, который выполняет следующие шаги\+:
\begin{DoxyEnumerate}
\item Устанавливает путь к компилятору g++ ({\ttfamily GPP}).
\item Задает исходные файлы проекта ({\ttfamily SOURCES}).
\item Задает опции компиляции ({\ttfamily CFLAGS}).
\item Компилирует исходные файлы в объектные файлы.
\item Компонует объектные файлы в исполняемый файл {\ttfamily XOgame.\+exe}.
\end{DoxyEnumerate}

Для запуска сборки проекта, необходимо выполнить указанный в {\ttfamily build.\+bat} скрипт, который автоматизирует процесс компиляции и линковки, создавая исполняемый файл {\ttfamily XOgame.\+exe}. 